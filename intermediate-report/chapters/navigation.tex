%TODO Edit this so it's not bad
\chapter{Using Sound to Navigate Space}
\section{Introduction to Intervals}
An \emph{interval} is a ratio of frequencies expressed as $\frac{f_1}{f_2}$, we're attempting to
generate these. We are using the \emph{just-intonation} system which is simply creating a system of
tuning using these whole-number ratios. A $p$-limit is a system where you limit your intervals to
those which prime-factorisation's highest prime is $p$. 5-limit is probably the first tuning system
theorised about in the West, based on Pythagoras' experiments in tuning.

Instruments in the west tend to be tuned using a system called 12-EDO, EDO meaning Equal Division of
Octave. The motivation for using these is that every interval consists only of members of the
harmonic series, and thus is `perfectly in tune'; EDO arrived out of a problem in tuning in 5-limit
being the `pythagorean comma' essentially an error that results from the octave being larger than it
should be. EDO smooths this error out over the whole range making things sound `good enough'.

The important part here is that we can generate ratios of frequencies and that integer
ratios are `good' i.e.\ a part of the harmonic series and thus should sound consonant.

\section{In 2D}

\begin{center}
    \begin{tikzpicture}
        \draw[help lines, color=gray!30](-2.9, -2.9) grid (2.9, 2.9);
        \draw[->, thick] (-3,0) -- (3,0) node[right]{$x$};
        \draw[->, thick] (0,-3) -- (0,3) node[above]{$y$};
    \end{tikzpicture}
\end{center}

Given two directions $x$ and $y$, we can create a ket-vector $|*\, x\, y \rangle$ which represents
$2^n 3^x 5^y$, where $n$ is any number, since there will be an equivalent interval possible i.e.
$|1\, 0\, 0\rangle$ and $|2\, 0\, 0\rangle$ are $2$ and $4$ respectively but this corresponds to an
octave (i.e.\ doubling the pitch), equally we can halve the interval until it's within the octave we
choose. 

As an example $3^1 \cdot 5^1 = 15$ which is a very large interval but doesn't make much sense as it
can be reduced to $\frac{15}{8}$ (major 7th) by adding a term $2^{-3}$ which brings it back into the
same octave. The interval is still a major 7th between octaves but some level of normalisation
needs to be present for it to be within the same octave. These names, slightly confusingly come from
the number of staff positions they take up (when normalised to one octave). As a rule we want the
interval to look like $1 \leq \frac{f_1}{f_2} < 2$, if the interval is less than $1$ multiply by $2$
if it is greater than $2$ then divide by $2$.

\subsection{Example}
For all of the examples, establish a base note, for ease assume something like $A4 = 440\si{\hertz}
= f_1$. Perhaps this could change with a some sort of generative compositional element independent
from the actual image on screen.

Now move in the space somehow $x=1$, $y=-1$.

\begin{center}
    \begin{tikzpicture}
        \draw[help lines, color=gray!30](-2.9, -2.9) grid (2.9, 2.9);
        \draw[->, thick] (-3,0) -- (3,0) node[right]{$x$};
        \draw[->, thick] (0,-3)-- (0,3) node[above]{$y$};

        \fill (1,-1) circle[radius=2pt];
    \end{tikzpicture}
\end{center}

This is represented in the vector $|0\, 1\, -1\rangle$ which corresponds to $2^0 3^1 5^{-1} =
\frac{6}{5}$. $\frac{6}{5}$ is a minor-third and thus the gap is 3 `semitones'. We should expect a
'$C$'. Quotes here because we're not talking about the 12-EDO system, instead just intonation.
5-limit just happens approximate the intervals well (partly because of the history of how the
western tuning system was developed).

\begin{align*}
    \frac{6}{5} \cdot f_1 &= f_2\\
    \frac{6}{5} \cdot 440\si{\hertz} &= 528\si{\hertz}
\end{align*}

Or in general we can do this:
\begin{align*}
    2^n \cdot 3^x \cdot 5^y \cdot f_1 &= f_2\\
\end{align*}

Thus if we get non-integer numbers for $x$ and $y$ we can calculate the frequencies too.
\begin{center}
    \begin{tikzpicture}
        \draw[help lines, color=gray!30](-2.9, -2.9) grid (2.9, 2.9);
        \draw[->, thick] (-3,0) -- (3,0) node[right]{$x$};
        \draw[->, thick] (0,-3)-- (0,3) node[above]{$y$};

        \fill (1,-0.75) circle[radius=2pt];
    \end{tikzpicture}
\end{center}

Where $x=1$ and $y=0.75$ we can simply find the frequency as $2^0 \cdot 3^1 \cdot 5^{-0.75} \cdot
440\si{\hertz} = 394.772\si{\hertz}$. This will sound dissonant and not particularly nice, perhaps
the image may look disordered.

The idea here is simple: tones will be consonant in some `focal point' and dissonant in-between,
hopefully this will allow there to be `touchstone' points in a space that allow them to be
navigated. These will be the points on some grid where the parameters are integers.

\section{In Higher Dimensions}
This idea extends into higher dimensions naturally, simply adding to the number of terms in the
vector $|*\, e_3\, e_5\, \cdots\, e_p \rangle$ which then gives the ratio as $2^n \cdot 3^{e_3}
\cdot 5^{e_5} \cdots p^{e_p}$ where $p$ is the $p$-th prime number. Each of these exponents could
represent some parameter added to the program. It's worth nothing, here too, this would be called
$p$-limit tuning.

\subsection{3D example}
Now we use 7-limit tuning with a vector like $| *\, x\, y\, z\rangle$

\begin{center}
    \begin{tikzpicture}
        \draw[->, thick] (-3,0,0) -- (3,0,0) node[right]{$x$};
        \draw[->, thick] (0,-3,0) -- (0,3,0) node[above]{$y$};
        \draw[->, thick] (0,0,-3) -- (0,0,3) node[above]{$z$};

        \fill (1,1,-1) circle[radius=2pt] node[right]{$(1,1,-1)$};
    \end{tikzpicture}
\end{center}

Thus we get a value of $3^1 \dot 5^1 \cdot 7^-1 \cdot 440\si{\hertz} = 942\si{\hertz}$ which we can
normalise back into the octave with $2^{-1}$ if we wish this then becomes $\frac{15}{14}$ which is
a small interval.

\section{Drawbacks}
The immediate drawback I can see is that as more parameters are added the less `strongly consonant'
integer-values of the parameters would be. This is mostly a cultural thing, as we are not used to
higher-limit tuning, but it's also worth noting that every interval possible in a limit below the
currently chosen limit is possible so adding more parameters doesn't shut off the possibility of
only exploring one or two of them at a time, creating more obviously consonant intervals.
