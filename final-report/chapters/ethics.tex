\chapter{Ethics}
This project does not handle any personal data, nor deals with identification of
people from data. There is no subject matter that may offend or otherwise effect
groups of people. 

However there is a question over ownership and who owns images produced by this
program if they are indistinguishable from Viner's own work.  One point to
consider is that where his work is pen-plotter images on listing paper, this is
purely computer generated imagery. 

Of course, also the provenance associated with the work is also completely
different; in the case of Prado's Mona List for example, it's clear that despite
being made around the same time, in the same workshop, and of the same subject
it is a separate piece of work with a different story. This perhaps is not the
best example however the creator of the Prado version likely worked in da
Vinci's studio at the time \citep{museodelprado_MonaLisa}. 

It's also worth mentioning the fair dealing exception to UK copyright law which
says that you are allowed to ``copy limited extracts of work for non-commercial
research or private study"\citep{govuk_copyright}. However I am unsure if this
legal protection extends to \emph{the potential for generating} copies of work.
It is clear to me however that I am not attempting to reproduce directly and
instead explore the techniques of creation of this work using completely
different technology.
