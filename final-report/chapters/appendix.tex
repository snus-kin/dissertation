\chapter{External Materials}
%TODO Write this better
\begin{itemize}
    \item Libraries
    \begin{itemize}
        \item \verb|p5.js| javascript library: \url{https://p5js.org/}
        \item \verb|tone.js| javascript library: \url{https://tonejs.github.io/}
        \item \verb|Data-Tree| javascript library:
            \url{http://cchandurkar.github.io/Data-Tree/}
    \end{itemize}
    \item Other External Materials
    \begin{itemize}
        \item Some of the initial graphics code was based on notes by my
            supervisor, John Stell
        \item The code for the additive synth was based on and heavily modified
            from \url{https://github.com/ejarzo/additive-synth}
    \end{itemize}
\end{itemize}

\chapter{Audio Demo}
\label{audiodemo}
\begin{lstlisting}[language=java]
import processing.sound.*;

SinOsc base;
SinOsc variable;
float baseFreq = 440;
// they are in unity to start with
float variableFreq = 440;

float volume = 0.2;
float s_x,s_y,x,y = 0;
float interval;

boolean isUp, isDown, isLeft, isRight;

void settings() {
  size(501, 501);
}

void setup() {  
  base = new SinOsc(this);
  base.play();
  base.freq(baseFreq);
  base.amp(volume);

  variable = new SinOsc(this);
  variable.play();
  variable.amp(volume);
}

void draw() {
  background(255);
  translate(width/2, height/2);
  scale(1, -1);

  // make some axes
  stroke(0);
  strokeWeight(1);
  line(0,height/2,0,-height/2);
  line(width/2,0,-height,0);

  strokeWeight(4);
  point(x,y);

  // mark where the integer values will be, for demo reasons
  stroke(255,0,0);
  strokeWeight(4);
  for(int i = -3; i <= 3; i++) {
    for(int j = -3; j <= 3; j++) {
      point(map(i, -3, 3, -width/2, width/2),
            map(j, -3, 3, -height/2, height/2));
    }
  }

  // scale x and y to some reasonable numbers -3 -> 3 
  s_x = map(x, -width/2, width/2, -3, 3);
  s_y = map(y, -height/2, height/2, -3, 3);
  // now we need to find a way to scale until 
  // we're in the same octave
  interval = pow(3,s_x) * pow(5,s_y);
  
  while (interval < 1 || interval >= 2) {
    if (interval < 1) {
      interval = interval * 2;
    } else if (interval > 2) {
      interval = interval / 2;
    }
  }

  variableFreq = interval * baseFreq;
  variable.freq(variableFreq);

  if (isUp) y++;
  if (isDown) y--;
  if (isRight) x++;
  if (isLeft) x--;
}

void keyPressed() {
  setMove(key, true);
}

void keyReleased() {
  setMove(key, false);
}

void setMove(char k, boolean b) {
  if (key == 'w') isUp = b;
  if (key == 's') isDown = b;
  if (key == 'a') isLeft = b;
  if (key == 'd') isRight = b;
}
\end{lstlisting}

