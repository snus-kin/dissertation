This project explores the work of Darrell Viner and produces a program that
allows users to explore the space his pen-plotter work inhabits. To do this a
variety of techniques are developed and deployed. Also, methods of exploring
this space using sound and music are explored. 

The project both provides potential users with a pleasant experience and a
greater understanding of Viner's work, and provide techniques that can be
applied to other computer art and music projects.

The techniques created in this project include: a point-wise method for
approximating the rounding of Perlin noise; a method of using audio to demarcate
space in higher-dimensions; and using Markov chains to generate melodies.
